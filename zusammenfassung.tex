\documentclass{scrartcl}
\thispagestyle{empty}
\usepackage{geometry}
\geometry{top=20mm, left=30mm, right=30mm, bottom=20mm}
\usepackage[utf8]{inputenc}
\usepackage[ngerman]{babel}
\usepackage[onehalfspacing]{setspace}
\RequirePackage{garamondx}
\begin{document}
\begin{center}
    \LARGE \textbf{Zusammenfassung} \\
    \vspace{.1in}
    \large der Bachelorarbeit \\
    \vspace{.1in}
    \small ,,Theorie und Evaluation des Feedreaders gylfeed" \\
    \vspace{.1in}
    \small \textit{Susanne Kießling, \today}
    \vspace{.2in}
\end{center}


Innerhalb der Projektarbeit \cite{kiessling} fand der Entwurf und die
Implementierung des Feedreaders \textit{gylfeed} statt. Es handelt sich um einen
Desktop-Feedreader, der vorallem in der Bedienung und dem Aufbau der grafischen
Benutzeroberfläche eine Alternative zu bestehenden Feedreadern anstrebt.
Entwickelt wurde der Feedreader in der Programmiersprache Python und dem Tool
zur Erstellung grafischer Benutzeroberflächen, GTK+.
\\
\\
Die Bachelorarbeit betrachtet die theoretischen Grundlagen und Hintergründe 
der Projektarbeit \cite{kiessling}. Statistische Auswertungen prüfen diese 
theoretischen Ansätze. Die jeweilige Umsetzung innerhalb von \textit{}gylfeed 
wird vorgestellt und bewertet.
\\
\\
Anfangs wird die Verwendung von sogenannten Signalen betrachtet und ein
Überblick zum Feedreader gylfeed gegeben. Der Hauptteil der Bachelorarbeit
beschäftigt sich mit der Beschaffung und Verarbeitung der Feed-Daten. Bei der
Beschaffung der Feed-Daten wird die Performance des Downloads thematisiert. Dazu
werden der synchrone und der asynchrone Ansatz im Vergleich betrachtet und jeweils
ein Performance-Test durchgeführt. Außerdem wird mit den Attributen ETag und
last-modified eine Möglichkeit erläutert, festzustellen, ob sich die Feed-Daten
geändert haben und ein Download notwendig ist. Dazu werden Stichprobentests
durchgeführt. 
\\
\\
Innerhalb der Verarbeitung der Feed-Daten wird anfangs anhand einer
Stichprobe die Häufigkeit verschiedener Feedformate untersucht. Danach wird
näher auf die Inhaltselemente der Feed-Daten eingegangen und untersucht, wie
häufig die einzelnen Elemente vorkommen. In beiden Bereichen, sowohl Beschaffung
als auch Verarbeitung der Feed-Daten wird die Umsetzung innerhalb \textit{gylfeed}
vorgestellt und mögliche Verbesserungen aufgeführt.
\\
\\
Auf weiterführende Konzepte wird im letzten Kapitel eingegangen. Die Erweiterung
der Suche mittels verschiedener Algorithmen wird vorgestellt. Als zweite
mögliche Erweiterung für \textit{gylfeed} wird der Einbezug von Nutzer-Präferenzen
betrachtet.
\\
\\
Abschließend erfolgt eine Zusammenfassung der behandelten Themen und gewonnenen
Ergebnisse.
\\
\\
\vspace*{-2em}\large\textbf{Literatur}
\renewcommand*{\refname}{} 
\vspace*{-1em}
\small
\bibliographystyle{unsrt}
\bibliography{refs.bib}
\end{document}
